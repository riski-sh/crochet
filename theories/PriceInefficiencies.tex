\documentclass{article}
\title{Simple Market Micro Structure Inefficiencies}
\author{washcloth}
\begin{document}

  \maketitle

  As traders we are aware two main concepts, price action and volume. Some
  say that these are the most important tools a trader has in hand since they
  represent rawly what the market is doing, and who is in control. This paper
  suggests a theory to the change of price in relation to volume of a financial
  market. Looking at a market through an intraday lense can show that
  the amount a volume needed to move a price toward a specific price increases
  exponentially as the day goes on until the market closes, in which case no
  amount of volume can move the price.
  \\

  This theory is very easy to model. Let  us define a function to represent
  cumulative intraday volume $v(c)$.

  \[
    v_c(t) = \sum_{n=0}^{t} v(t)
  \]

  where $v(t)$ is the closing volume at candle $t$, $t = 0$ is the first candle
  of the opening day, and $0 \le t \le 330$ since there are $330$ minutes in a
  normal trading day. For clarity, the function $p(t)$ is the closing price
  of candle $t$.

  Now, we show that a function exists with inputs price $p$, and volume $v$,
  that decays to $p$ as $v \rightarrow \infty$.

  \[
    f(t) = \frac{v_c(t)}{\frac{v_c(t)}{p(t)} - \frac{p(t)}{v_c{t}}}
  \]

  As an exersize to the reader, show the limit of $f(t)$ is indeed $p(t)$.
  How is this useful in anyway? Well, one can take advantage of these small
  micro inefficiences once market conditions are correct.
  There are many constrants where this function is not useful,
  for example when $p(t) = v_c(t)$, which happens at most once a day.
  When $v_c(t) \le p(t)$, this function becomes a poor model since it produces
  a negative price, therefore constraning the useful range even more. What
  is the useful range of using this model? The useful range of this model
  is completly dependent of the value of $\frac{p(t)}{v_c(t)}$, since
  the value of $\frac{p(t)}{v_c(t)}$ determines how much volume is needed
  to push the price of a security in any given direction. Unfortuantly
  there is no easy answer to when this will perform as intended. I encourage
  you to explore the use of this function in helping you pick securities
  and take advantage of micro inefficiencies.

\end{document}
